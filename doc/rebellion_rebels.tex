%*
%* Seven Kingdoms: Ancient Adversaries
%*
%* Copyright 1997,1998 Enlight Software Ltd.
%* Copyright 2018 Timothy Rink
%*
%* This program is free software: you can redistribute it and/or modify
%* it under the terms of the GNU General Public License as published by
%* the Free Software Foundation, either version 2 of the License, or
%* (at your option) any later version.
%*
%* This program is distributed in the hope that it will be useful,
%* but WITHOUT ANY WARRANTY; without even the implied warranty of
%* MERCHANTABILITY or FITNESS FOR A PARTICULAR PURPOSE.  See the
%* GNU General Public License for more details.
%*
%* You should have received a copy of the GNU General Public License
%* along with this program.  If not, see <http://www.gnu.org/licenses/>.
%*
%*

\chapter{Rebellion and Rebels}

Rebellion is something that you must constantly be on the lookout for. Rebellion can transform a mighty army into a dangerous foe in seconds. It can cause large areas of your Empire to become neutral or even actively hostile.

Below is a list of possible situations concerning Rebels.

\section{Rebellious Units}

If the loyalty level of one of your units falls below the 30 point mark, it may betray your Kingdom and change its color to that of one of your foes

\section{Rebellious Peasants}

Peasants in your Villages may on occasion rebel when more than two-thirds of a Village’s loyalty to you drops below the 30 point level. You must, therefore, be very careful when recruiting your Villagers. If you try to recruit when their loyalty is 30 percent or below, they will rebel.

\section{Mobile Rebels}

Mobile Rebels may attack the Villages where they originated, dispatching those still loyal to your rule.

They may attack your firms.
They may move off and settle a new Village.
They may move into a Village belonging to another Rebel group.

\section{Rebel Controlled Villages}

Rebels may take control of Villages, making them independent. A Rebel Controlled Village will be labeled “Controlled by Rebels” on the Village interface.

Even though you have built and staffed a Fort Linked to a Rebel Controlled Village, the resistance of the Villagers will not decrease.

These Villages may at times become new Kingdoms---with the rebel leader becoming the new King.

On the other hand, they may sometimes tire of their rebellion and revert to a normal Independent Village.

\section{Attacking Rebel Controlled Villages}

\index{attacking!rebel villages} When you attack a Rebel Village, the Rebels will come out to defend their town.

\section{Destroying a Rebel Village}

\index{destroying rebel villages}

If you destroy a Rebel Village, the Rebels will once again become mobile.